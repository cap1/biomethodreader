\documentclass{article}
\usepackage[utf8]{inputenc}
\usepackage[ngerman]{babel}
\usepackage{enumitem}
\usepackage{graphicx}
\usepackage{amsmath}
\usepackage[version=3]{mhchem} 		%% \ce{CO2} -- automatische subscript der zahlen
\usepackage{booktabs}
\usepackage{subfig}
\usepackage{textcomp}
\usepackage{multirow}
\usepackage{fixltx2e}
\usepackage{natbib}
\usepackage{color}

\begin{document}
\title{Vorlesung Protein und AS Umsatz}
\author{Christian Müller}
%\date{13.01.2012}

\subsection{Proteinumsatz und AS-Katabolismus}

\begin{itemize}
	\item Abbau: Proteine \textrightarrow AS 
	\item Daher sind Proteine aus der Nahrung ein lebenswichtige Quelle,
		zusätzlich aus defekten Proteinen
	\\ \hline
	\item Abbau zelluärer Proteine erfolgt mit unterschiedlicher Geschwindigkeit,
		Stabilität haengt an den N-Terminalen AS \textrightarrow ``N-Terminus Regel''
	\item Daten stammen von der Hefe
		\\ \hline
	\item Proteinumsatz unterliegt strenger Regulation
	\item ständig Prüfung und Markierung von Abbzubauenden Proteinen durch \textsl{Ubiquitin} (76As)
	\item Auch Regulatorische Funktionen bei ``einfacher Ubiquitinilierung''
	\item Mehrfache Ubiquitinilierung führt zu sofortigem Abbau, an AS 48
	\\ \hline
	\item Drei Enzyme beteligt an Ubiquitinkonjugation
		\begin{enumerate}
			\item U. ativierendes E1
			\item U. ativierendes E2
			\item Ubiqitin-Protein-Ligase E3 ``Adapter-Moleküle'', katalysier Spezifität
		\end{enumerate}
	\item Viele Varianten für E3, 800 Gene bei Arabidopsis
	\item Ketten aus 4-5 Ubiquitin besonders determinieren für Abbau
	\item \textrightarrow Ubiquitin-System Hauptunterschied von Tieren zu Pflanzen
	\\ \hline
	\item Proteasomen verdauen mit Ubiquitin markierte Proteine
	\item Katalytisches Zentrum ist das 20S-Proteasom aus 28UE, ATP abhängig
	\item 19S-UEs binden spezifische Ubiquitinketten
	\\ \hline
	\item Proteinabbaun regiuliert auch diverse biologische Funktionen
	\textrightarrow Transkription, Immunsystem, Zellzyklus\dots
	\\ \hline
	\item Übersicht über den AS-Katabolismus bie Säugetieren (Bild)
	\item Erster Schritt bei AS-Abbau immer die die Abspaltung der Aminogruppe %TODO srsly?
	\item $$\alpha$$-Aminogruppen werden durch oxidative Desaminierung von Glutaat in Ammoniumionen überführt
		\textrightarrow Daher Harnstoffzyklus wichtig zur Entgiftung
	\\ \hline
	\item Die Erste Reaktion wird durch AS Spezifische Transmainasen kataklysiert,
		Reaktion ist reversibel und kann auch zur Synthese genutzt werden
	\item Glutamat-DH, in den Mitochondrien, zum Schutz vor toxischen Ammoniak, RedOx-Reaktion
	\item bei Wirbeltieren erfolgt Umwandlung des Ammoniums in der Leber zu Harnstoff und wird auch als 
		dieser ausgeschieden
	\\ \hline
	\item Transaminasereadktion, PLP
	\item Chiralität, nur L-Aminosäuren
	\item Schaubilder Umsatz: Reilreaktionen
		Wichtig: Welcher Schritt sorgt für Chiralität bei AS?
		\\ \hline 
	\item Serin und Threonin können direkt desaminiert werden, katalysier durch Serin/Threonin-DH
\end{itemize}

\end{document}
