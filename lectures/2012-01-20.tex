\documentclass{article}
\usepackage[utf8]{inputenc}
\usepackage[ngerman]{babel}
\usepackage{enumitem}
\usepackage{graphicx}
\usepackage{amsmath}
\usepackage[version=3]{mhchem} 		%% \ce{CO2} -- automatische subscript der zahlen
\usepackage{booktabs}
\usepackage{subfig}
\usepackage{textcomp}
\usepackage{multirow}
\usepackage{fixltx2e}
\usepackage{natbib}
\usepackage{color}

\begin{document}
\title{Vorlesung Aminosäuresynthse}
\author{Christian Müller}
%\date{13.01.2012}

\subsection{Biosynthese der AS}

\begin{itemize}
	\item Stickstoffzyklus
	\item Stickstofffixierung benötigt hochwirksamen Katakylsator da sehr inert
	\item Haber-Bosch und Rhizobien als Symbionten an Wurzeln
	\item \texrightarrow Nitrogenase-Komplex aus Zwei UE muss gekapselt werden,
		da es sonst \ce{02} reduzieren würde
	\item Leghämoglobin bindet freien Sauerstoff in den Zellen
	\item Aufbau und Funktion des Nitrogenasekomplexes
	\item \dots
	\item Biosynthese der AS, 4 Familien
	\item Verlust langer Synthesewege
	\item Shikimat und Chorismat malen können
	\item Ansatzpunkt für Herbizide \textrightarrow Gylphosate (Roundup)
	\item Synthese von Phenylalanin bzw. Tyrosin
	\item Tryptophan längster Syntheseweg, ausgehend von Chorismat
	\item Regulation der Synthesewege Komplex durch Verzweigung
	\item Oft Rückkopplungshemmung von Endprodukt und Zwischenprodukte
	\item Biosynthese von Porphyrine, als Ausgangspunkt für Hämoglobin/``dieses PS Molekül''
\end{itemize}
\end{document}
