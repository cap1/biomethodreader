\section{Nucleotides}
\subsection{Basics}
\begin{table}[h!]
	\centering
	\begin{tabular}{l l l}
		\toprule
		Basename		&	Basetype		& Info \\
		\midrule 
		Adenine		&	Purine		&	\\
		Cytosine		&	Pyrimidine	&	\\
		Guanine		&	Purine		&	\\
		Thymine		&	Pyrimidine	&	\\
		Urcail		&	Pyrimidine	&	\\
		\bottomrule
	\end{tabular}
	\caption{Basic Information on Nucleotides}
	\label{tab:nucleotides}
\end{table}

\subsection{Mutagenesis}

Random mutagenesis can be induced in several ways,
Table \ref{tab:mutagenesis} shows some possebilities.

\begin{table}
	\centering
	\begin{tabular}{l l}
		\toprule
		\multirow{2}{*}{Pysical mutagenesis}	&	ionic radiation \\
															&	UV \\		
		\midrule
		\multirow{3}{*}{Chemical mutagenesis}	&	EMS, Ethylmethano sulfonate \\
															&	ENU, Ethylnitro urea \\		
															&	5-bromine-uracil \\		
		\midrule
		\multirow{2}{*}{PCR Methods}				&	DNA shuffling \\
															&	error prone PCR \\		
		\midrule
		\multirow{2}{*}{Insertional mutagenesis}	&	transposon mutagenenesis \\
																&	signature tagged mutagenesis (STM) \\		
																&	T-DNA mutagenesis \\
		\bottomrule
	\end{tabular}
	\caption{Random Mutagenensis}
	\label{tab:mutagenesis}
\end{table}

To induce a site directed mutagenesis the follwing methods can be used.

\begin{description}
	\item[Deletion or insertion at restriction site] \hfill \\
	\item[Insertion of mutated oligonucleotides] \hfill \\
	\item[PCR] \hfill \\
	\item[Insertional mutagenesis, homologous recombination] \hfill \\
\end{description}

